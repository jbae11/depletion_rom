\begin{abstract}

    \gls{NFC} simulations are system-level analyses that track
    material flow in a \gls{NFC}. 

    Since \gls{NFC} simulations
    involve many facilities, the fidelity of modeling each
    facility is sacrificed for quickness. 

    One of the most important functionalities in a \gls{NFC}
    simulation is the depletion of nuclear fuel in a reactor,
    which is directly related to the \gls{UNF} composition
    and waste / material profile.

    Depletion calculations for fuel cycle simulations in current
    \gls{NFC} simulators are either
    recipe based, [so on and so forth, with citation].

    We trained a set of regression algorithms to predict the
    \gls{UNF} composition of a \gls{PWR} assembly, given
    burnup and enrichment. The training sample used is the  Unified
    database, which is a part of the \gls{UNF-STANDARDS} tool developed
    by Peterson et al. \cite{peterson_used_2013}.

    For each isotope, we pick the regression algorithm with the smallest root mean squared error for the testing dataset, and store the
    set of trained algorithms in a pickled python file. 

    We tailored a training algorithm to each isotope, in order to
    better predict the relationship between burnup, enrichment,
    and isotope composition (because isotopes have varying relationships
    with burnup and enrichment).

    We then validate the set of trained algorithms in two ways -
    to predict the U.S. \gls{UNF} inventory profile in 2020,
    and predict SERPENT depletion calculation results.

    1.The model approach predicts \gls{UNF} inventory far better than
    the recipe approach, and takes into account the varying burnup and
    enrichment.
    2. The model approach can predict the composition of future
    \gls{UNF} with higher burnup and enrichment [andrei]

\end{abstract}
