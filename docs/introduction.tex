\section{Introduction}



\subsection{Background and Motivation}
The \gls{NFC} is a complex system of facilities and material 
mass flows that are combined to provide nuclear energy for use 
in society, usually in the form of electricity 
\cite{yacout_modeling_2005}. 
\gls{NFC} simulators are system analysis tools used to investigate 
issues related to the dynamics of a nuclear fuel cycle in both 
high and low resolution. 
An example of a high resolution element is the spent fuel 
isotopic composition from a single fuel bundle, and an example 
of a low resolution element is the total fuel utilization in 
the system. 
The intention behind the use of \gls{NFC} simulators is to develop 
a better understanding of the dependence between various components 
in the system and the effects of changes on the system. 
Their goal is to assist in evaluating and improving potential 
strategies for nuclear power development in terms of improving waste 
management, economic competitiveness, etc \cite{yacout_modeling_2005}.   

One of the major functionalities of a \gls{NFC} simulator is its 
ability to transmute nuclear fuel in a reactor based on reactor 
conditions such as burnup, enrichment, etc. 
The transmutation results impact the accuracy of the \gls{UNF} 
composition, and thus the capability of using the \gls{NFC} to 
analyze the impact of a \gls{NFC} on variables such as waste profile.  

Current fuel cycle simulator codes include \Cyclus, ORION, DYMOND, 
and VISION. 
Each of these NFC simulators obtains transmutation results using 
different methods. 
Some of these codes have multiple options for obtaining 
transmutation results. 
There are essentially two overall types of methods to obtain transmutation 
results: recipe-based method and library-based method. 
In the recipe-based method, direct neutronics calculations are not performed 
within the model, it is done externally \cite{yacout_vision_2006}. 
The resulting fresh and spent fuel compositions for specific parameters 
are entered directly into the \gls{NFC} model \cite{sunny_transition_2015}. 
In the library-based method,  the \gls{NFC} code dynamically calculates depleted 
fuel recipes by interpolating reactor data libraries generated by
reactor physics burnup calculation code such as SERPENT 
\cite{leppanen_serpent_2013}, \gls{ORIGEN} \cite{croff_users_1980}, and etc. 
Table \ref{tab:nfcs} shows a simple breakdown of the 
transmutation methods available for each \gls{NFC} code. 

\begin{table}[h]
    \centering
    \begin{tabular}{lrrr}
        \hline
        \gls{NFC} code & Transmutation Methods Available \\
        \hline
        \Cyclus & Recipe-based and Library-based\\
        ORION & Recipe-based and Library-based\\
        DYMOND & Recipe-based  \\
        VISION & Recipe-based  \\
        \hline
    \end{tabular}
    \caption{Methods for transmutation in reactor modules
             for each \gls{NFC} simulation code. 
             \label{tab:nfcs}}
\end{table}

\Cyclus has three methods to obtain transmutation results: the \Cycamore 
recipe reactor \cite{huff_extensions_2014}, CyBORG 
\cite{skutnik_cyborg_2016}, and Bright-lite \cite{flanagan_brightlite_2014}.  
The \Cycamore recipe reactor accepts a fresh and spent fuel recipe that is 
defined by the user. 
CyBORG calls the \gls{ORIGEN} solver \gls{API} directly to calculate depleted fuel
compositions, which then becomes the output fuel composition.
Bright-lite uses libraries that represent a specific reactor design to 
calculate the burnup, and output isotopic vectors for a given input fuel
\cite{flanagan_brightlite_2014}. 
ORION has a recipe-based and library-based method to obtain transmutation results. 
In ORION's library-based method, burnup-dependent cross section libraries 
for multiple reactor types with multiple initial fuel enrichment are 
generated before ORION analysis \cite{sunny_transition_2015}. 
The libraries are then used to generate spent fuel recipes based on 
reactor conditions such as burnup, enrichment, etc.  
DYMOND has a recipe-based method where recipes for both the input 
and output fuel compositions are externally calculated. 
Isotopes are tracked in lumped categories: fission products, minor 
actinides, Uranium and Plutonium \cite{feng_standardized_2016}.  
VISION is an extension of the DYMOND code. 
It has a similar recipe-based method as the DYMOND code, but with individual isotopic 
tracking \cite{yacout_vision_2006}. 

For \gls{NFC} simulators, striking a balance between fidelity 
and computational cost is a key issue. 
The advantage of using high fidelity models is its inherent flexibility 
to readily accommodate varying fuel compositions that are present 
when modeling complex scenarios that involve isotopic changes
\cite{sunny_transition_2015}. 
However, using high fidelity models for century-long simulations 
can result in impractical computational times. 
The advantage of using low fidelity models (recipe-based method)
is the low computational cost. 
It is an acceptable method for modeling fuel cycles with fixed input 
composition or fuel cycles at equilibrium \cite{sunny_transition_2015}. 
However, for fuel cycles not at equilibrium, it can result in less 
accurate results. 

Therefore, to find a middle ground of accurate depletion data while 
maintaining a practical computational cost, this paper introduces 
a trained neural network model that is able to predict \gls{PWR} \gls{UNF}
composition based on initial enrichment and burnup. 

\subsection{\Cyclus}

\Cyclus is an agent-based nuclear fuel cycle simulation framework 
\cite{huff_fundamental_2016}, meaning
that each reactor and fuel cycle facility is modeled as a discrete and independent
player in the simulation.
A \Cyclus agent archetype defines the logic that governs the behavior
of an agent. 
\Cyclus archetypes can be coded either with \CC  or python.
In this simulation, the user defines the archetype's
parameters. The archetypes with user-defined parameters are then deployed
as agent prototypes.  Encapsulating the \texttt{Facility} agents are the \texttt{Institution} and \texttt{Region}.
A \texttt{Region} agent holds a set of \texttt{Institution}s. 
An \texttt{Institution} agent can deploy or decommission \texttt{Facility} agents.

At each timestep,
agents make requests for materials or bid to supply them and exchange
with one another. A market-like mechanism called the dynamic resource exchange
\cite{gidden_methodology_2016} governs the exchanges.
For output analysis, each material resource has a quantity, composition, name, and a unique identifier.

Cyclus has multiple advantages over other available
\gls{NFC} simulation codes including open-source distribution, modularity,
and extensibility. Its agent-based modeling approach
is ideal for modeling coupled, physics-dependent
supply chain problems common in \glspl{NFC}.
The framework allows for dynamic loading of 
external libraries, which allows the users to plug-and-play
different types of physics models for \gls{NFC}
simulation.


\subsubsection{Modularity and Extensibility}

In most modern \glspl{NFC} simulators, the facilities and their
behaviors (and their fidelities) are confined in the software.
Also, most modern \gls{NFC} simulators model
fuel cycles (once-through, continuous reprocessing)
with immutable connections between facilities. On the
other hand, \Cyclus allows users to plug-and-play various agent models
within the \Cyclus framework (shown in figure \ref{fig:core}).
Also, \Cyclus relies on a market-based model
for material trades between facilities, so the user can design
any novel fuel cycle. This enables \Cyclus to simulate any system analysis
involving multiple connected facilities with physics-based
calculations.

\begin{figure}[htbp!]
    \begin{center}
        \includegraphics[width=\textwidth]{cyclus_core.png}
    \end{center}
    \caption{The \Cyclus core provides APIs that the archetypes
            can be loaded into the simulation modularly
            \cite{huff_fundamental_2016}.}
    \label{fig:core}
\end{figure}

Due to this modularity in the \Cyclus framework, the developed
model in this work can be implemented independently without
having to modify the \Cyclus source code. The new facility archetype
is written in Python and implemented through the \Cyclus API.
