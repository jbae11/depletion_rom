\section{Introduction}
\subsection{Background and Motivation}
Problem definition

=========================GWEN=========================
\begin{itemize}
    \item definition of NFC
    \item Quickness vs accuracy
    \item Depletion calculations for LWRs
    \item current practice for depletion calculations [citations]
    \item pros and cons of those practices [citations]
    \item Does it even matter? Fuel cycle simulation is a system-level analysis. How much does it matter? [citations]
\end{itemize}


\gls{NFC} simulations are system-level analyses that track
material flow in a \gls{NFC}. 

Since \gls{NFC} simulations
involve many facilities, the fidelity and detail of modeling each
facility are sacrificed for quickness. 

One of the most crucial functionalities in a \gls{NFC}
simulation is the transmutation of nuclear fuel in a reactor,
which is directly related to the \gls{UNF} composition
and waste / material profile.

Depletion calculations for fuel cycle simulations in current
\gls{NFC} simulators are either
recipe based (no calculation performed), 
library based, [so on and so forth, with citation].



1.The model approach predicts \gls{UNF} inventory far better than
the recipe approach, and takes into account the varying burnup and
enrichment.
2. The model approach can predict the composition of future
\gls{UNF} with higher burnup and enrichment [andrei]

=========================GWENEND=========================

\subsection{\Cyclus}

\Cyclus is an agent-based nuclear fuel cycle simulation framework 
\cite{huff_fundamental_2016}, meaning
that each reactor and fuel cycle facility is modeled as a discrete and independent
player in the simulation.
A \Cyclus agent archetype defines the logic that governs the behavior
of an agent. In this simulation, the user defines the archetype's
parameters. The archetypes with user-defined parameters are then deployed
as agent prototypes.  Encapsulating the \texttt{Facility} agents are the \texttt{Institution} and \texttt{Region}.
A \texttt{Region} agent holds a set of \texttt{Institution}s. 
An \texttt{Institution} agent can deploy or decommission \texttt{Facility} agents.

Several versions of \texttt{Institution}
and \texttt{Region} agents exist, varying in complexity and purpose \cite{huff_extensions_2014}.
\texttt{DeployInst}, which deploys agents at user-defined timesteps, serves
as the main \texttt{Institution} archetype in this work. All reactor \texttt{Facility} agents,
fuel reprocessing, and fabrication \texttt{Facility} agents
are deployed through \texttt{DeployInst}, while basic fuel cycle \texttt{Facility} agents
such as sink, source, enrichment, and storage facilities are deployed 
through \texttt{NullInst}, which simply deploys \texttt{Facility}
agents at the beginning of the simulation.

At each timestep,
agents make requests for materials or bid to supply them and exchange
with one another. A market-like mechanism called the dynamic resource exchange
\cite{gidden_methodology_2016} governs the exchanges.
For output analysis, each material resource has a quantity, composition, name, and a unique identifier.

In this work, each nation is represented as a \texttt{Region} agent,
that contains \texttt{Institution} agents, which deploy \texttt{Facility} 
agents according to a user-defined deployment scheme.

Cyclus has multiple advantages over other available
\gls{NFC} simulation codes including open-source distribution, modularity,
and extensibility. Its agent-based modeling approach
is ideal for modeling coupled, physics-dependent
supply chain problems common in \glspl{NFC}.
The framework allows for dynamic loading of 
external libraries, which allows the users to plug-and-play
different types of physics models for \gls{NFC}
simulation.


\subsubsection{Modularity and Extensibility}

In most modern \glspl{NFC} simulators, the facilities and their
behaviors (and their fidelities) are confined in the software.
Also, most modern \gls{NFC} simulators model
fuel cycles (once-through, continuous reprocessing)
with immutable connections between facilities. On the
other hand, \Cyclus allows users to plug-and-play various agent models
within the \Cyclus framework (shown in figure \ref{fig:core}).
Also, \Cyclus relies on a market-based model
for material trades between facilities, so the user can design
any novel fuel cycle. This enables \Cyclus to simulate any system analysis
involving multiple connected facilities with physics-based
calculations.


\begin{figure}[htbp!]
    \begin{center}
        \includegraphics[width=\textwidth]{cyclus_core.png}
    \end{center}
    \caption{The \Cyclus core provides APIs that the archetypes
            can be loaded into the simulation modularly
            \cite{huff_fundamental_2016}.}
    \label{fig:core}
\end{figure}

Within the \Cyclus kernel, the \gls{DRE} connects
the framework and the agents by mediating agent material
offers and requests.
The kernel solves the multicommodity exchange problem
posed by the material offers and requests and executes
the transaction between two agents.

\Cyclus archetypes could also be coded with Python,
allowing the created model to be plugged into the archetype.
