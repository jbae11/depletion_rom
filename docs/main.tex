\documentclass[review]{elsarticle}

\usepackage{multirow}
\usepackage{lineno}
\usepackage{xspace}
\usepackage{threeparttable}
\usepackage{subfig}
\modulolinenumbers[5]

\journal{Progress in Nuclear Energy}

%% `Elsevier LaTeX' style
\bibliographystyle{elsarticle-num}
%%%%%%%%%%%%%%%%%%%%%%%

%%%% packages and definitions (optional)
\usepackage{placeins}
\usepackage{booktabs} % nice rules (thick lines) for tables
\usepackage{microtype} % improves typography for PDF
\usepackage{hhline}
\usepackage{amsmath}
\usepackage{subfig}

%\usepackage[demo]{graphicx}
%\usepackage{caption}
%\usepackage{subcaption}

\usepackage{booktabs}
\usepackage{threeparttable, tablefootnote}
\graphicspath{ {../images/} }

\usepackage{tabularx}
\newcolumntype{b}{>{\hsize=1.0\hsize}X}
\newcolumntype{s}{>{\hsize=.5\hsize}X}
\newcolumntype{m}{>{\hsize=.75\hsize}X}
\newcolumntype{x}{>{\hsize=.25\hsize}X}

\newcommand{\Cyclus}{\textsc{Cyclus}\xspace}%
\newcommand{\Cycamore}{\textsc{Cycamore}\xspace}%

% tikz %
\usepackage{tikz}
\usetikzlibrary{positioning, arrows, decorations, shapes}

\usetikzlibrary{shapes.geometric,arrows}
\tikzstyle{process} = [rectangle, rounded corners, minimum width=3cm, minimum height=1cm,text centered, draw=black, fill=blue!30]
\tikzstyle{object} = [ellipse, rounded corners, minimum width=3cm, minimum height=1cm,text centered, draw=black, fill=green!30]
\tikzstyle{arrow} = [thick,->,>=stealth]

% hyperref %
\usepackage[hidelinks]{hyperref}
% after hyperref %
\usepackage{cleveref}
\usepackage{datatool}
\usepackage[acronym,toc]{glossaries}
\include{acros}

\makeglossaries

\begin{document}
\begin{frontmatter}
\title{Neural Network Model Applications to Reactor Fuel Depletion for Fuel Cycle Modeling}

\date{}                     %% if you don't need date to appear

%% Authors
\author{Jin Whan Bae}
\author[uiuc]{Andrei Rykhlevskii}
\author[uiuc]{Gwendolyn Chee}
\author[uiuc]{Kathryn D. Huff\corref{corrauthor}}
\cortext[corrauthor]{Corresponding Author}
\ead{kdhuff@illinois.edu}


% Institutes of the authors
\input{institutions}

\begin{keyword}
nuclear fuel cycle \sep
machine learning \sep
neural network \sep
simulation \sep
spent nuclear fuel \sep
\end{keyword}

\begin{abstract}

We trained a neural network model
to predict \gls{PWR} \gls{UNF} composition given initial enrichment
and burnup, to implement a quick, flexible, medium-fidelity method to
estimate depleted \gls{PWR} fuel assembly composition
to model scenarios where the \gls{PWR} fuel burnup
and enrichment varies over time.
The \gls{UNF-STANDARDS} \gls{UDB} provided a ground truth on which
the model trained.
We validated the model by comparing the U.S. \gls{UNF} inventory profile
predicted by the model with the \gls{UDB} \gls{UNF} inventory
profile. The neural network yields less than 1\% error of
\gls{UNF} inventory decay heat and activity and less than
2\% error for major isotopic inventory. The neural network model
takes 0.27 seconds for 100 depletion calculations, compared to
the 118 seconds for 100 \gls{ORIGEN} calculations.

We also implemented this model into \Cyclus, an agent-based
\gls{NFC} simulator, to perform rapid, medium-fidelity
\gls{PWR}
depletion calculations. This model also allows discharge of
batches with varying burnup assemblies.

Since the original private data cannot be retrieved from the model,
this trained model allows open-source depletion capabilities to \gls{NFC}
simulators. This method of training an artificial neural network with a
dataset from a complex fuel depletion model
can provide rapid, medium-fidelity depletion capabilities
to large-scale fuel cycle simulations.

\end{abstract}

\end{frontmatter}
\glsresetall

\linenumbers
\section{Introduction}
\subsection{Background and Motivation}
Problem definition

=========================GWEN=========================

The \gls{NFC} is a complex system of facilities and material 
mass flows that are combined to provide nuclear energy for use 
in society, usually in the form of electricity 
\cite{yacout_modeling_2005}. 
\gls{NFC} simulators are system analysis tools used to investigate 
issues related to the dynamics of a nuclear fuel cycle in both 
high and low resolution. 
An example of a high resolution element is the spent fuel 
isotopic composition from a single fuel bundle, and an example 
of a low resolution element is the total fuel utilization in 
the system. 
The intention behind the use of \gls{NFC} simulators is to develop 
a better understanding of the dependence between various components 
in the system and the effects of changes on the system. 
Their goal is to assist in evaluating and improving potential 
strategies for nuclear power development in terms of improving waste 
management, economic competitiveness, etc \cite{yacout_modeling_2005}.   

One of the major functionalities of a \gls{NFC} simulator is its 
ability to transmute nuclear fuel in a reactor based on reactor 
conditions such as burnup, enrichment, etc. 
The transmutation results impact the accuracy of the \gls{UNF} 
composition, and thus the capability of using the \gls{NFC} to 
analyze the impact of a \gls{NFC} on variables such as waste profile.  

Current fuel cycle simulator codes include \Cyclus, ORION, DYMOND 
and VISION. 
Each of these NFC simulators obtain transmutation results using 
different methods. 
Some of these codes have multiple methods for obtaining 
transmutation results. 
Table \cite{tab:nfc_code} shows a simple breakdown of the 
transmutation methods available for each \gls{NFC} code. 

\begin{table}[h]
    \centering
    \label{tab:nfc_code}
    \begin{tabular}{lrrr}
        \hline
        \gls{NFC} code & Transmutation Methods Available \\
        \hline
        \Cyclus & Recipe and Library\\
        ORION & Recipe and Library\\
        DYMOND & Recipe \\
        VISION & Recipe \\
        \hline
    \end{tabular}
    \caption{\gls{NFC} Methods for transmutation in reactor module}
\end{table}

\Cyclus has three methods to obtain transmutation results. 

ORION 

DYMOND 

VISION 

There is essentially two overall types of methods to obtain transmutation 
results: recipe method and library method. 
The recipe method is a low fidelity method, while the library method is 
a high fidelity method. 
For \gls{NFC} simulators, striking a balance between fidelity 
and computational cost is a key issue. 
Using high fidelity models for century-long simulations 
can result in impractical computational times; while using low fidelity 
models can result in less accurate results. 




\begin{itemize}
    \item definition of NFC
    \item Quickness vs accuracy
    \item Depletion calculations for LWRs
    \item current practice for depletion calculations [citations]
    \item pros and cons of those practices [citations]
    \item Does it even matter? Fuel cycle simulation is a system-level analysis. How much does it matter? [citations]
\end{itemize}


\gls{NFC} simulations are system-level analyses that track
material flow in a \gls{NFC}. 

Since \gls{NFC} simulations
involve many facilities, the fidelity and detail of modeling each
facility are sacrificed for quickness. 



Depletion calculations for fuel cycle simulations in current
\gls{NFC} simulators are either
recipe based (no calculation performed), 
library based, [so on and so forth, with citation].



1.The model approach predicts \gls{UNF} inventory far better than
the recipe approach, and takes into account the varying burnup and
enrichment.
2. The model approach can predict the composition of future
\gls{UNF} with higher burnup and enrichment [andrei]

=========================GWENEND=========================

\subsection{\Cyclus}

\Cyclus is an agent-based nuclear fuel cycle simulation framework 
\cite{huff_fundamental_2016}, meaning
that each reactor and fuel cycle facility is modeled as a discrete and independent
player in the simulation.
A \Cyclus agent archetype defines the logic that governs the behavior
of an agent. 
\Cyclus archetypes can be coded either with c++ or python.
In this simulation, the user defines the archetype's
parameters. The archetypes with user-defined parameters are then deployed
as agent prototypes.  Encapsulating the \texttt{Facility} agents are the \texttt{Institution} and \texttt{Region}.
A \texttt{Region} agent holds a set of \texttt{Institution}s. 
An \texttt{Institution} agent can deploy or decommission \texttt{Facility} agents.

At each timestep,
agents make requests for materials or bid to supply them and exchange
with one another. A market-like mechanism called the dynamic resource exchange
\cite{gidden_methodology_2016} governs the exchanges.
For output analysis, each material resource has a quantity, composition, name, and a unique identifier.

Cyclus has multiple advantages over other available
\gls{NFC} simulation codes including open-source distribution, modularity,
and extensibility. Its agent-based modeling approach
is ideal for modeling coupled, physics-dependent
supply chain problems common in \glspl{NFC}.
The framework allows for dynamic loading of 
external libraries, which allows the users to plug-and-play
different types of physics models for \gls{NFC}
simulation.


\subsubsection{Modularity and Extensibility}

In most modern \glspl{NFC} simulators, the facilities and their
behaviors (and their fidelities) are confined in the software.
Also, most modern \gls{NFC} simulators model
fuel cycles (once-through, continuous reprocessing)
with immutable connections between facilities. On the
other hand, \Cyclus allows users to plug-and-play various agent models
within the \Cyclus framework (shown in figure \ref{fig:core}).
Also, \Cyclus relies on a market-based model
for material trades between facilities, so the user can design
any novel fuel cycle. This enables \Cyclus to simulate any system analysis
involving multiple connected facilities with physics-based
calculations.


\begin{figure}[htbp!]
    \begin{center}
        \includegraphics[width=\textwidth]{cyclus_core.png}
    \end{center}
    \caption{The \Cyclus core provides APIs that the archetypes
            can be loaded into the simulation modularly
            \cite{huff_fundamental_2016}.}
    \label{fig:core}
\end{figure}

Due to this modularity in the \Cyclus framework, the developed
model in this work can be implemented independently without
having to modify the \Cyclus source code. The new facility archetype
is simply written with the \Cyclus API, and imported in a
\Cyclus simulation.

\subsubsection{\Cycamore Recipe Reactor}
\Cycamore is a library that consists of useful
fuel cycle facility archetypes for \Cyclus. The
recipe reactor in \Cycamore is a batch-wise reactor.
A reactor core is consisted of multiple batches,
and a batch is consisted of multiple assemblies.
At startup, the reactor requests the entire core,
which is consisted of user-defined number of assemblies.
At every cycle, the reactor discharges and
requests batches of fuel. Upon decommissioning,
the reactor discharges all its fuel. The discharged
fuel is transmutated to the user-defined recipe.
No depletion calculation is performed. However,
the user can define multiple recipes, and can define
times to change the recipe from one to another.

\section{Method}

This work follows four steps:

\begin{enumerate}
\item Data curation
\item Model training (with hyperparameter optimization)
\item Model validation
\item Model implementation in \Cyclus.
\end{enumerate}


First, we curated the
data with assembly information for ease of use in training the model.
Second, we trained the neural network using Keras \cite{collet_keras_2015}
and Scikit-learn \cite{pedregosa_scikit-learn_2011},
while using an outer loop to
search for the optimized set of neural network hyperparameters,
such as the number of hidden layers and nodes per layer.
Third, we used the model to predict the U.S. \gls{UNF}
inventory as specified in the \gls{UDB} and compare
\gls{UNF} inventory metrics such as fissile content
and decay heat. Lastly, we implemented the trained
model in \Cyclus by developing a reactor facility archetype
that transmutes fuel using the model.

The files used to generate and test the neural network
model are all on Github (https://github.com/jbae11/depletion\_rom)
including the pickled neural network model file. The raw data
is not available to the public.

\subsection{Training set}

In order to train a artificial neural network model, a significant database of
depletion data is needed, that span the potential
burnup and enrichment range in typical reactors.

Data from the \gls{UDB} was used to train the model 
based on burnup and enrichment. 

Ideally, the data should be generated with exact same reactor
parameters other than burnup and enrichment, such as geometry or
coolant density. Also, it should be noted that the reactor
parameters could be input values to the neural network model. However, since
this is preliminary work to prove potential workflow, we
simply used all the \gls{PWR} datasets in the \gls{UDB},
and the input values are only burnup and initial enrichment.

\subsubsection{Unified Database}
The \gls{UDB} is part of a larger engineering
analysis tool, the \gls{UNF-STANDARDS}, developed
by \gls{ORNL} \cite{peterson_used_2013}.  
The database provides a comprehensive, controlled
source of \gls{SNF} information, including
dry cask attributes, assembly data, economic attributes,
transportation infrastructure attributes, potential future
facility attributes, and federal government radioactive
waste attributes. 
The assembly-specific attributes include
initial enrichment, burnup, \gls{MTHM}, assembly 
type and discharge date \cite{peterson_fuel_2015}. 
To generate this database, the authors used
irradiation and decay calculations using SCALE 
\cite{bowman_scale_2011}. The calulations were performed on each 
spent fuel assembly based on the previously mentioned 
parameters in the collected data to obtain mass, heat 
and activity for each assembly \cite{peterson_additional_2017}. 
All the assemblies were modeled with conservative 
depletion parameters which result in the hardening of 
the neutron energy spectrum and an increased \gls{SNF} 
residual reactivity \cite{peterson_additional_2017}. 

Also, the irradiation history of the fuel is unspecified in the
database, which can be a source of deviation for short-lived isotopes.
With the unknown parameters (unknown irradiation history, variyng
assembly models) and assumptions (conservative composition to
increase fuel reactivity), the database is far from ideal to use
as a training dataset for a depletion calculation model. However,
we chose this data set because it allows testing of model performance
through comparision of \gls{UNF}
inventory between a high-fidelity model and a model prediction for
varying burnup, enrichment, and discharge time.

Ideally, the data should be generated with exact same reactor
parameters other than burnup and enrichment, such as geometry or
coolant density. However, since
this is preliminary work to prove potential workflow, we
simply used all the \gls{PWR} datasets in the \gls{UDB}.

We received the database through personal contact with
Kaushik Banerjee, one of the original creators of the database.
The database is currently accessible via \gls{RSICC}.

\subsection{Data Curation}

We curated the raw \gls{UDB} datasets to generate
a cleaner training set. First we only used the 
\gls{PWR} assemblies since \gls{BWR} \gls{UNF} assembly
calculation results can vary significantly with void fraction.
We also filtered out the
`very low' enrichment ($\leq$ 1.5) and
burnup ($\leq$ 10,000 MWd/MT)
assemblies to represent a more modern \gls{PWR} \gls{UNF}
assembly range. Figure \ref{fig:enr_bu} shows the
burnup and enrichment distribution of the assemblies in the
\gls{UDB}.


\begin{figure}
    \centering
    \includegraphics[width=\textwidth]{enr_bu.png}
    \caption{Burnup and enrichment distribution of train
             datasets curated from the \gls{UDB}.}
    \label{fig:enr_bu}
\end{figure}


Also, the SCALE calculation in the \gls{UDB} only tracks 60 isotopes
and on average 3.5 weight percent of the \gls{UNF} is not accounted for. We
aggregate the isotopes not accounted for as a separate category. Lastly,
we processed the database so that the isotopic compositions are 
represented as weight \% from initial uranium mass, to normalize
the dataset. For every isotope \textit{i}:

\begin{equation}
x_i = \frac{m_i}{M_{initU}}
\end{equation}
where:
\[
x_i = \text{\% weight of isotope in depleted assembly}
\]
\[
m_i = \text{mass of isotope in depleted assembly in \gls{UDB}}
\]
\[
M_{ITHM} = \text{Mass of initial uranium in assembly}
\]


\subsection{Predictive Models for Fuel Depletion}

\gls{UNF} depleted composition is difficult to predict
due to the varying relationship with the fuel parameters.
In figures \ref{fig:cs_137, fig:pu_239, fig:u_235},
we plotted isotopic concentration values against
burnup and enrichment, to observe the relationship between
burnup, enrichment, and isotopic concentration.

We observed that the isotopic concentration has varying
relationships with fuel burnup and enrichment.
For example, if the isotopic population is mainly determined by
the fission of initial uranium, a linear regression algorithm
can be used to predict the isotopic concentration from burnup
(Cs-137 shown in figure \ref{fig:cs_137}).
However, isotopes like plutonium-239 (figure \ref{fig:pu_239}) have multiple, conflicting creation
and destruction terms, making it harder to predict using a
linear regression algorithm. Also, the uranium-235 (figure \ref{fig:u_235}) concentration
depends on both burnup and enrichment, which can make it
hard to predict using a simple linear model.

Due to these complexities, we decided to train an artificial
neural network for our predictive model. We chose
Keras \cite{collet_keras_2015} to create and validate the model,
as well as scikit-learn \cite{pedregosa_scikit-learn_2011}
and pandas \cite{mckinney-proc-scipy-2010} for data processing and management.

\begin{figure}
    \centering
    \includegraphics[width=\textwidth]{cs-137_sub.png}
    \caption{$^{137}Cs$ concentration in a \gls{UNF} assembly
             varies linearly with assembly burnup.}
    \label{fig:cs_137}
\end{figure}

\begin{figure}
    \centering
    \includegraphics[width=\textwidth]{pu-239_sub.png}
    \caption{$^{239}Pu$ concentration in a \gls{UNF} assembly
             is not linearly related to burnup, since it
             is affectex by multiple, conflicting creation
             and destruction terms}
    \label{fig:pu_239}
\end{figure}


\begin{figure}
    \centering
    \includegraphics[width=\textwidth]{u-235_sub.png}
    \caption{$^{235}U$ concentration in a \gls{UNF} assembly
             somewhat proportional to both enrichment and
             burnup, but is difficult to predict using
             a simple linear regression algorithm.}
    \label{fig:u_235}
\end{figure}


\subsection{Training and Selecting Models}

The inputs (features) of the model are
burnup (MWd/MT) and initial enrichment (wt\% $^{235}U$).
The outputs (targets) of the model are
the concentration (weight \%) of the 60 isotopes in the
depleted assembly.

With the curated dataset, we performed an outer loop
search to find the best-performing neural network
hyperparameter. First, we set 20 percent of the 
data for final model testing purpose. We used a three-fold
cross validation \cite{stone1974cross} on the remaining
dataset, to
measure the average prediction error value. We
normalized the data using the sklearn MinMaxScaler
so that the range of input and output data is (0,1).

\begin{table}[h]
    \centering
    \begin{tabular}{lr}
        \hline
        Parameter & Values \\
        \hline
        Number of hidden layers & 1, \textbf{2}, 3, 4 \\
        Nodes per hidden layer & 4, 16, 32, 64, \textbf{128} \\
        Dropout rate & \textbf{0.0}, 0.2, 0.5 \\
        \hline
    \end{tabular}
    \caption{Table of hyperparameters tested
             for neural network model. The bold
             numbers are the values we used for the final model.}
\end{table}


We selected the model with the smallest average error value
and exported the model as a file using python
pickle file, along with the dataset normalization objects, and 
the list of isotopes. By exporting the trained model
as a self-contained file, the model can be used in any python
application.


\subsection{Model Testing}

We tested the accuracy of the model by comparing
the model's \gls{UNF} composition prediction
in three different cases. First, we compared the
isotope-by-isotope prediction error of the model for an
assembly with a specific burnup and enrichment.
Second, we compared the waste characteristics of
an assembly for all assemblies. Third, we compared
the predicted total \gls{PWR} \gls{UNF} inventory with
\gls{UDB}. The metric for error is calculated as
relative error percentage, calculated by:
\begin{equation}
\epsilon = \frac{x_{data} - x_{model}}{x_{data}}
\end{equation}

This metric is to provide a fair assessment to errors
in predicting isotopes with minute concentration in \gls{UNF}.
However, it should be noted that a large error percentage in the
prediction of these minute isotopes is mostly because it has been
divided by a small value, not because the absolute error is large.

We compared parameters of the \gls{UNF} inventory
such as activity and decay heat, using the
\gls{PyNE} \cite{scopatz_pyne:_2012}. This tool provides
various functions used
in this work, such as decaying and calculating
the activity and decay heat of the \gls{UNF} inventories.

Ideally, the model should be tested against data
that is not part of the training data. However, given
that the purpose of this model is to allow accessible and
quick depletion calculation for
fuel cycle simulations, the model would suffice if
it predicts the dataset well. In other words, the
goal of the model is to be able to reproduce,
in a continuous manner, the range of fuel depletion
calculations in the database without access to the particular dataset.
This work
is to propose a general concept for implementing
rapid depletion models in fuel cycle simulations.
For creating depletion models for other reactor designs or depletion parameters,
one would simply change the dataset to a set of depletion calculations performed
for a specific reactor design and operational parameters.

We also tested the neural network's performance with the
test set that was set aside prior to the hyperparameter search.



\subsection{Model Implementation in \Cyclus}

The trained model is exported to a file
that can be plugged into external codes. \Cyclus
has a python interface that allows the developer
to design archetypes in python. We developed a reactor
module that behaves similarly to the recipe reactor
but calculates depleted fuel concentration using the
imported model instead of a recipe. The user can vary
individual assembly burnups. The user defines a burnup
and enrichment
matrix for the reactor. The rows are the number
of batches, and the columns are the number of
assemblies in a batch. This reactor module is also
available on Github 
(https://github.com/jbae11/ann\_pwr).

This sort of implementation can be done with
a recipe-based approach for modeling reactor depletion
if the user defines multiple
output recipes. However, the user can only define
the recipe of a batch.
Implementing this trained neural network model will allow the user to vary
burnup and enrichment for individual assemblies, as well
as vary fuel residence time and burnup with reactor
lifetime or simulation time. Such capability will be
useful in simulating \gls{UNF} inventory in the future,
where the burnup of \gls{LWR} fuel will increase
with advanced fuel technology.
\section{Results}

The model performed better than using the average
recipe in predicting the U.S. \gls{UNF}, with
negligible increase in computational time.

\subsection{Depletion Calculation Time and File Size}
For 100 random sets of
burnup and enrichment depletion predictions,
the model takes 0.27 seconds to output discharge composition, while
searching the database for assemblies
with the closest burnup and enrichment (using Pandas)
takes 21.8 seconds. Comparatively, 100 \gls{ORIGEN} calculations
take 118 seconds. Using the model achieves 43,700\% reduction
in time and does not require libraries, or a reactor physics code.
The standalone model pickle file is only
38 Kb, while the curated database (.csv) is 330 Mb.

\subsection{Assembly Comparison}

Ten data points were randomly sampled from the \gls{UDB},
and were compared with the model predictions to observe
two things:
\begin{enumerate}
    \item What isotopes the model is good / bad
        at predicting
    \item What burnup / enrichment range the model is good / bad
        at predicting
\end{enumerate}

Figures \ref{fig:3-2_29998-0}, \ref{fig:3-81_35883-0},
\ref{fig:4-0_35195-0}, and \ref{fig:4-47_50105-0}
show that the model
generally has a high relative error percentage for \textsuperscript{226}Ra
(average concentration $6.0\times10^{-12}\%$),
\textsuperscript{227}Ac (average concentration  $2.3\times10^{-12}\%$), and curium isotopes.
The absolute prediction errors are quite small
(averaging $1e-11$), but the large percent errors are due
to the small value of the data. There was not a notable
difference in the error values for enrichment
and burnup variations.

\begin{figure}
    \centering
    \includegraphics[width=\textwidth]{3-2_29998-0.png}
    \caption{Isotopic composition prediction error for an assembly with 
             $29.998 \frac{GWd}{MT}$ burnup and 3.2  \% enrichment.}
    \label{fig:3-2_29998-0}
\end{figure}

\begin{figure}
    \centering
    \includegraphics[width=\textwidth]{3-81_35883-0.png}
    \caption{Isotopic composition prediction error for an assembly with 
             $35.883 \frac{GWd}{MT}$ burnup and 3.81  \% enrichment.}
    \label{fig:3-81_35883-0}
\end{figure}

\begin{figure}
    \centering
    \includegraphics[width=\textwidth]{4-0_35195-0.png}
    \caption{Isotopic composition prediction error for an assembly with 
             $35.193 \frac{GWd}{MT}$ burnup and 4.0 \% enrichment.}
    \label{fig:4-0_35195-0}
\end{figure}


\begin{figure}
    \centering
    \includegraphics[width=\textwidth]{4-47_50105-0.png}
    \caption{Isotopic composition prediction error for an assembly with 
             $50.105 {GWd}{MT}$ burnup and 4.47  \% enrichment.}
    \label{fig:4-47_50105-0}
\end{figure}

\FloatBarrier

\subsection{U.S. \gls{UNF} Inventory Comparison}

In this section, we compare three \gls{UNF} inventory composition
model approaches.
The only difference is the composition of the
assemblies. The three different inventory compositions were acquired by:

\begin{enumerate}
    \item \textbf{Data}: directly query the assembly composition from the \gls{UDB}.
    \item \textbf{Prediction}: neural network prediction of depleted composition using burnup and enrichment from database
    \item \textbf{Recipe}: using a single composition (recipe) for all assemblies. Assumes all compositions are the same.
\end{enumerate}

The median values for burnup and initial enrichment are
$41,552$ MWd/MT and 3.85 (wt\%), respectively. The concentrations of major
isotopes in the assembly are in Table \ref{tab:avg_assem}.


\begin{table}[h]
    \centering
    \begin{tabular}{|l|r|r|r|r|r|}
        \hline
        Isotopes & $^{235}U$ & $^{238}U$ & $^{239}Pu$ & $^{137}Cs$ & $^{90}Sr$ \\
        \hline
        Concentration [wt\%] & 1.076 & 92.66 & 0.77 & 0.14 & 0.061 \\
        \hline
    \end{tabular}
    \caption{Isotopic concentration of the assembly with median burnup and
             enrichment. This composition is used for the recipe method. 
    \label{tab:avg_assem}}
\end{table}


We compare the three composition predictions according to:
\begin{enumerate}
    \item Isotopic inventory
    \item Waste metrics (activity and decay heat)
    \item Equivalent fissile inventory (equivalent $^{239}Pu$)
\end{enumerate}

The \gls{UDB} contains discharged assembly data
from nuclear reactors in the United States up to May of
2013. We added all the \gls{UNF} assemblies in the database
and evaluated the inventory as it was in 2013. 
Table \ref{tab:met} shows the comparison of the inventories.

\begin{table}[h]
    \centering
    \begin{tabular}{l|r|rr}
        \hline
        Metric & Data & Recipe & Prediction \\
        \hline
        $^{239}Pu$ mass [t] & 320.37 & 351.70 & 321.38\\
        $^{137}Cs$ mass [t] & 63.84 & 66.64 & 63.73 \\
        $^{235}U$ mass [t] & 464.60 & 487.94 & 474.14\\
        $^{238}U$ mass [t] & 42,171 & 42,016 & 42,162\\
        \hline
        Decay Heat [MW] & 193.39 & 198.55 & 193.33 \\
        Activity [$E+21$Bq] & $2.79$ & $2.84$ & $2.75$ \\
        \hline
    \end{tabular}
    \caption{Comparison of \gls{PWR} \gls{UNF} inventory in the U.S,
             obtained from direct data query, recipe approach,
             and neural network prediction. 
    \label{tab:met}}
\end{table}

\FloatBarrier

\subsubsection{Isotopic Inventory}

In terms of isotopic composition accuracy, the trained
neural network model outperforms the
mean recipe method for all isotopes.
Figure \ref{fig:iso_rel} shows the relative
error between the full database, model prediction, and
the mean recipe for
major isotopes. For plutonium isotopes, the trained neural
network model far
outperforms the mean database.

\begin{figure}
    \centering
    \includegraphics[width=\textwidth]{iso_rel.png}
    \caption{Neural network model prediction error relative to median
             \gls{UDB} recipe, for key isotopes.}
    \label{fig:iso_rel}
\end{figure}


\begin{figure}
    \centering
    \includegraphics[width=\textwidth]{pu_rel.png}
    \caption{Neural network model prediction error relative to median
             \gls{UDB} recipe, for plutonium isotopes.}
    \label{fig:pu_rel}
\end{figure}

\FloatBarrier


\subsubsection{Waste Management Metrics}
The trained neural network excellently predicts the activity
and decay heat metrics. Figures \ref{fig:assem_dh} and \ref{fig:assem_act}
show the relative error percent of the decay heat and activity
predictions per assembly. The model predicts 99.5\% of
assemblies with an error of less than 1\%.
Figures \ref{fig:assem_dh_recipe} and
\ref{fig:assem_act_recipe} show the relative error
of the decay heat and activity calculated with the average
recipe method.
Unsurprisingly, the error increases as the actual burnup and enrichments
diverge from the average.

\begin{figure}
    \centering
    \includegraphics[width=\textwidth]{assem_dh.png}
    \caption{Relative error percentage for predicting the decay
             heat of individual assemblies.}
    \label{fig:assem_dh}
\end{figure}


\begin{figure}
    \centering
    \includegraphics[width=\textwidth]{assem_act.png}
    \caption{Relative error percentage for predicting the
             activity of individual assemblies.}
    \label{fig:assem_act}
\end{figure}



\begin{figure}
    \centering
    \includegraphics[width=\textwidth]{assem_dh_recipe.png}
    \caption{Relative error in decay heat calculated by the average recipe
             method. The red point is the median enrichment and
             burnup.}
    \label{fig:assem_dh_recipe}
\end{figure}

\begin{figure}
    \centering
    \includegraphics[width=\textwidth]{assem_act_recipe.png}
    \caption{Relative error in activity
             calculated by the average recipe method.
             The red point is the median enrichment and
             burnup.}
    \label{fig:assem_act_recipe}
\end{figure}

\FloatBarrier


Table \ref{tab:wm} shows the decay heat and activity
comparison in the years 2020, 2100, and 3100. The total
error is less than 1.1\% for all metrics at all time periods.
Figure \ref{fig:ha_err} shows relative error in activity and
decay heat as a function of time. It shows
that the model outperforms the average recipe method
in predicting waste metrics.

%! Do the decimal centering
\begin{table}[h]
    \centering
    \begin{tabular}{lcrrr}
        \hline
        Metric & Year & UDB [MW] & Prediction [MW]  & Error [\%] \\
        \hline
        \multirow{3}{*}{\shortstack{Decay \\ Heat }} & 2020 & 40.97 & 41.07 & -0.24 \\
                                                    & 2100 & 16.42 & 16.47 & -0.35 \\
                                                    & 3100 & 3.13 & 3.14 & -0.15 \\
        \hline
         & & UDB [$10^{19}$Bq] & Prediction [$10^{19}$Bq] & Error[\%] \\
        \multirow{3}{*}{\shortstack{Activity}} & 2020 & 46.70 & 46.60 & 0.21 \\
                                               & 2100 & 6.39& 6.38 & 0.07 \\
                                               & 3100 & 0.36 & 0.36 & -0.17 \\
        \hline
    \end{tabular}
    \caption{Decay heat and radioactivity values and errors for years 2020, 2100, and 3100.}
    \label{tab:wm}
\end{table}

\begin{figure}
    \centering
    \includegraphics[width=\textwidth]{ha_err.png}
    \caption{Relative error of waste management metrics for \gls{UNF} inventory
             generated by the average recipe and the prediction model.}
    \label{fig:ha_err}
\end{figure}

\FloatBarrier

\subsubsection{Assembly fissile quality}

Fissile quality is frequently quantified in units of 
$^{239}Pu$ equivalent, shown in figure \ref{tab:pu_equiv} \cite{anon_plutonium_1989}. This value is
calculated by aggregating the weighted fissile
values of each isotope in a material. The $^{239}Pu$
equivalent factors are different for fast neutron spectrum 
and thermal neutron spectrum reactors \cite{baker_comparison_1963}.  
For example, the equivalent fissile value for
an \gls{LWR} will be calculated by:
\begin{gather*}
\text{Pu}_{\text{eq}} = \sum_i w_i m_i \\
i \in [^{235}U, ^{238}Pu, ^{239}Pu, ^{240}Pu, ^{241}Pu, ^{242}Pu, ^{242}Am] \\
w_i = \text{equivalent weighting factors} \\
m_i = \text{mass of iso i} \\
\end{gather*}
Where the variables represent the mass of each isotope.



\begin{table}[h]
    \centering
    \begin{tabular}{ccc}
        \hline
        & \multirow{2}{*}{\shortstack{LWR \\ (Thermal)}} &
        \multirow{2}{*}{\shortstack{\gls{FBR}\\ (Fast)}} \\ \\
        \hline
        $^{235}U$ & +0.8& +0.8\\
        $^{238}Pu$ & -1.0& +0.44\\
        $^{239}Pu$ & +1.0& +1.0\\
        $^{240}Pu$ & -0.4& +0.14\\
        $^{241}Pu$ & +1.3& +1.5 \\
        $^{242}Pu$ & -1.4& +0.037\\
        $^{241}Am$ & -2.2& -0.33\\
        \hline
    \end{tabular}
    \caption{$^{239}Pu$ equivalence factors from \cite{anon_plutonium_1989}.
             Factors are separately reported for thermal and fast spectra.}
    \label{tab:pu_equiv}
\end{table}


Figure \ref{fig:fiss} shows the fast spectrum $^{239}Pu$ equivalent
value of the \gls{UNF} inventory plotted over time.
The trained model outperforms the recipe method. The
initial falls for all three lines are due to the
decay of plutonium 241, which has a half-life of
14 years.


\begin{figure}
    \centering
    \includegraphics[width=\textwidth]{fiss.png}
    \caption{$^{239}Pu$ equivalent value in time for three
             inventories. The model predictions match closely
             with the value from the database.}
    \label{fig:fiss}
\end{figure}




\subsection{\Cyclus implementation}

In this work, we trained a neural
network model and implemented it as a  \Cyclus reactor
agent that predicts \gls{UNF} composition. The model predicts spent
fuel composition
based on customizable reactor parameters such as
discharge burnup, initial enrichment, cycle time, and power
capacity. The created archetype in \Cyclus also allows users to define
time-dependent
equations instead of constants for reactor parameters.
The user can define an enrichment-burnup matrix for
each assembly in each batch, and the burnup and enrichment
values can be equations in time. This way, users can
implement reactor facilities in which the reactor parameters
change in time (e.g. to represent reactor uprates, industry
burnup trends, etc.).

Figures \ref{fig:cyclus_pu}
and \ref{fig:cyclus_fp} show the discharge fuel composition
of a reactor facility in which we increased the discharge burnup
from 33,000 to 71,710 MWd/MT over 25 discharge cycles.
It should be noted that the model does not take into account
the plausibility of such fuel depletion. For example, it
would be nearly impossible for a  \gls{PWR} to burn 2\%
enriched \gls{UOX} fuel to 70,000 MWd/MT.

The user can also define time-varying
cycle time and refueling time for the reactor model, as shown
in figure \ref{fig:cyclus_time}.


%! it is plutonium in the fuel total
\begin{figure}
    \centering
    \includegraphics[width=\textwidth]{./cyclus_imp/pu.png}
    \caption{Plutonium isotope composition in discharge fuel over discharge cycle. The model does not predict well for the target burnup values
    that are over the burnup listed in the training dataset.}
    \label{fig:cyclus_pu}
\end{figure}


\begin{figure}
    \centering
    \includegraphics[width=\textwidth]{./cyclus_imp/fp.png}
    \caption{Fission product concentration in discharge fuel over discharge cycle. Increased discharge burnup leads to higher fission product concentration.}
    \label{fig:cyclus_fp}
\end{figure}


\begin{figure}
    \centering
    \includegraphics[width=\textwidth]{./cyclus_imp/cycle_time.png}
    \caption{Discharge and refueling cycles can be defined as an equation of time in this reactor archetype. Discharge burnup is scaled to take into account longer fuel residence time, and leads to increase in discharge fuel \textsuperscript{244}Cm composition.
    }
    \label{fig:cyclus_time}
\end{figure}

\subsubsection{Applications and Use Cases}

The capability to set dynamic reactor parameters
allows simulation of various future transition scenarios
that depend on \gls{UNF} inventory characteristics,
such as \gls{MA} inventory. 
Users can simulate future scenarios in which the discharge
burnup of reactors increases over time to reveal
impacts on \gls{MA} inventory and, correspondingly,
transition speed. 

With
advances in materials, reactors may have longer
cycle times and higher fuel discharge burnups.
This dynamic reactor model will be able to account
for the changes in these reactor parameters.
Also, users can simulate potential
power uprates in currently existing fleets, and
estimate corresponding impacts on \gls{UNF} inventory.
\FloatBarrier
\section{Conclusion}

[plug about how it's `validated' against the trained dataset]
We cannot avoid the criticism that the model is validated
against the dataset used to train the model. However, the purpose
of this work is to create a model that can quickly reproduce the
database without having to distribute the database, which is proprietary
and large in size. For the purpose of having an accessible
depletion model, this method holds promise.


eh, it's not the best, and it takes a really long time to get them going,
but it's good enough, isn't it?

future work:
a single model to go from
burnup, enrichment -> composition matrix
perhaps a neural network? preliminary results
show that they are worse. 

\bibliography{bibliography}


\end{document}
