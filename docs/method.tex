\section{Method}

This work follows four steps:

\begin{enumerate}
\item Data curation
\item Model training (with hyperparameter optimization)
\item Model validation
\item Model implementation in \Cyclus.
\end{enumerate}


First, we curated the
data with assembly information for ease of use in training the model.
Second, we trained the neural network using Keras \cite{collet_keras_2015}
and Scikit-learn \cite{pedregosa_scikit-learn_2011},
while using an outer loop to
search for the optimized set of neural network hyperparameters,
such as the number of hidden layers and nodes per layer.
Third, we used the model to predict the U.S. \gls{UNF}
inventory as specified in the \gls{UDB} and compare
\gls{UNF} inventory metrics such as fissile content
and decay heat. Lastly, we implemented the trained
model in \Cyclus by developing a reactor facility archetype
that transmutes fuel using the model.

The files used to generate and test the neural network
model are all on Github (https://github.com/jbae11/depletion\_rom)
including the pickled neural network model file. The raw data
is not available to the public.

\subsection{Training set}

In order to train a artificial neural network model, a significant database of
depletion data is needed, that span the potential
burnup and enrichment range in typical reactors.

Data from the \gls{UDB} was used to train the model 
based on burnup and enrichment. 

Ideally, the data should be generated with exact same reactor
parameters other than burnup and enrichment, such as geometry or
coolant density. Also, it should be noted that the reactor
parameters could be input values to the neural network model. However, since
this is preliminary work to prove potential workflow, we
simply used all the \gls{PWR} datasets in the \gls{UDB},
and the input values are only burnup and initial enrichment.

\subsubsection{Unified Database}
The \gls{UDB} is part of a larger engineering
analysis tool, the \gls{UNF-STANDARDS}, developed
by \gls{ORNL} \cite{peterson_used_2013}.  
The database provides a comprehensive, controlled
source of \gls{SNF} information, including
dry cask attributes, assembly data, economic attributes,
transportation infrastructure attributes, potential future
facility attributes, and federal government radioactive
waste attributes. 
The assembly-specific attributes include
initial enrichment, burnup, \gls{MTHM}, assembly 
type and discharge date \cite{peterson_fuel_2015}. 
To generate this database, the authors used
irradiation and decay calculations using SCALE 
\cite{bowman_scale_2011}. The calulations were performed on each 
spent fuel assembly based on the previously mentioned 
parameters in the collected data to obtain mass, heat 
and activity for each assembly \cite{peterson_additional_2017}. 
All the assemblies were modeled with conservative 
depletion parameters which result in the hardening of 
the neutron energy spectrum and an increased \gls{SNF} 
residual reactivity \cite{peterson_additional_2017}. 

Also, the irradiation history of the fuel is unspecified in the
database, which can be a source of deviation for short-lived isotopes.
With the unknown parameters (unknown irradiation history, variyng
assembly models) and assumptions (conservative composition to
increase fuel reactivity), the database is far from ideal to use
as a training dataset for a depletion calculation model. However,
we chose this data set because it allows testing of model performance
through comparision of \gls{UNF}
inventory between a high-fidelity model and a model prediction for
varying burnup, enrichment, and discharge time.

Ideally, the data should be generated with exact same reactor
parameters other than burnup and enrichment, such as geometry or
coolant density. However, since
this is preliminary work to prove potential workflow, we
simply used all the \gls{PWR} datasets in the \gls{UDB}.

We received the database through personal contact with
Kaushik Banerjee, one of the original creators of the database.
The database is currently accessible via \gls{RSICC}.

\subsection{Data Curation}

We curated the raw \gls{UDB} datasets to generate
a cleaner training set. First we only used the 
\gls{PWR} assemblies since \gls{BWR} \gls{UNF} assembly
calculation results can vary significantly with void fraction.
We also filtered out the
`very low' enrichment ($\leq$ 1.5) and
burnup ($\leq$ 10,000 MWd/MT)
assemblies to represent a more modern \gls{PWR} \gls{UNF}
assembly range. Figure \ref{fig:enr_bu} shows the
burnup and enrichment distribution of the assemblies in the
\gls{UDB}.


\begin{figure}
    \centering
    \includegraphics[width=\textwidth]{enr_bu.png}
    \caption{Burnup and enrichment distribution of train
             datasets curated from the \gls{UDB}.}
    \label{fig:enr_bu}
\end{figure}


Also, the SCALE calculation in the \gls{UDB} only tracks 60 isotopes
and on average 3.5 weight percent of the \gls{UNF} is not accounted for. We
aggregate the isotopes not accounted for as a separate category. Lastly,
we processed the database so that the isotopic compositions are 
represented as weight \% from initial uranium mass, to normalize
the dataset. For every isotope \textit{i}:

\begin{equation}
x_i = \frac{m_i}{M_{initU}}
\end{equation}
where:
\[
x_i = \text{\% weight of isotope in depleted assembly}
\]
\[
m_i = \text{mass of isotope in depleted assembly in \gls{UDB}}
\]
\[
M_{ITHM} = \text{Mass of initial uranium in assembly}
\]


\subsection{Predictive Models for Fuel Depletion}

\gls{UNF} depleted composition is difficult to predict
due to the varying relationship with the fuel parameters.
In figures \ref{fig:cs_137, fig:pu_239, fig:u_235},
we plotted isotopic concentration values against
burnup and enrichment, to observe the relationship between
burnup, enrichment, and isotopic concentration.

We observed that the isotopic concentration has varying
relationships with fuel burnup and enrichment.
For example, if the isotopic population is mainly determined by
the fission of initial uranium, a linear regression algorithm
can be used to predict the isotopic concentration from burnup
(Cs-137 shown in figure \ref{fig:cs_137}).
However, isotopes like plutonium-239 (figure \ref{fig:pu_239}) have multiple, conflicting creation
and destruction terms, making it harder to predict using a
linear regression algorithm. Also, the uranium-235 (figure \ref{fig:u_235}) concentration
depends on both burnup and enrichment, which can make it
hard to predict using a simple linear model.

Due to these complexities, we decided to train an artificial
neural network for our predictive model. We chose
Keras \cite{collet_keras_2015} to create and validate the model,
as well as scikit-learn \cite{pedregosa_scikit-learn_2011}
and pandas \cite{mckinney-proc-scipy-2010} for data processing and management.

\begin{figure}
    \centering
    \includegraphics[width=\textwidth]{cs-137_sub.png}
    \caption{$^{137}Cs$ concentration in a \gls{UNF} assembly
             varies linearly with assembly burnup.}
    \label{fig:cs_137}
\end{figure}

\begin{figure}
    \centering
    \includegraphics[width=\textwidth]{pu-239_sub.png}
    \caption{$^{239}Pu$ concentration in a \gls{UNF} assembly
             is not linearly related to burnup, since it
             is affectex by multiple, conflicting creation
             and destruction terms}
    \label{fig:pu_239}
\end{figure}


\begin{figure}
    \centering
    \includegraphics[width=\textwidth]{u-235_sub.png}
    \caption{$^{235}U$ concentration in a \gls{UNF} assembly
             somewhat proportional to both enrichment and
             burnup, but is difficult to predict using
             a simple linear regression algorithm.}
    \label{fig:u_235}
\end{figure}


\subsection{Training and Selecting Models}

The inputs (features) of the model are
burnup (MWd/MT) and initial enrichment (wt\% $^{235}U$).
The outputs (targets) of the model are
the concentration (weight \%) of the 60 isotopes in the
depleted assembly.

With the curated dataset, we performed an outer loop
search to find the best-performing neural network
hyperparameter. First, we set 20 percent of the 
data for final model testing purpose. We used a three-fold
cross validation \cite{stone1974cross} on the remaining
dataset, to
measure the average prediction error value. We
normalized the data using the sklearn MinMaxScaler
so that the range of input and output data is (0,1).

\begin{table}[h]
    \centering
    \begin{tabular}{lr}
        \hline
        Parameter & Values \\
        \hline
        Number of hidden layers & 1, \textbf{2}, 3, 4 \\
        Nodes per hidden layer & 4, 16, 32, 64, \textbf{128} \\
        Dropout rate & \textbf{0.0}, 0.2, 0.5 \\
        \hline
    \end{tabular}
    \caption{Table of hyperparameters tested
             for neural network model. The bold
             numbers are the values we used for the final model.}
\end{table}


We selected the model with the smallest average error value
and exported the model as a file using python
pickle file, along with the dataset normalization objects, and 
the list of isotopes. By exporting the trained model
as a self-contained file, the model can be used in any python
application.


\subsection{Model Testing}

We tested the accuracy of the model by comparing
the model's \gls{UNF} composition prediction
in three different cases. First, we compared the
isotope-by-isotope prediction error of the model for an
assembly with a specific burnup and enrichment.
Second, we compared the waste characteristics of
an assembly for all assemblies. Third, we compared
the predicted total \gls{PWR} \gls{UNF} inventory with
\gls{UDB}. The metric for error is calculated as
relative error percentage, calculated by:
\begin{equation}
\epsilon = \frac{x_{data} - x_{model}}{x_{data}}
\end{equation}

This metric is to provide a fair assessment to errors
in predicting isotopes with minute concentration in \gls{UNF}.
However, it should be noted that a large error percentage in the
prediction of these minute isotopes is mostly because it has been
divided by a small value, not because the absolute error is large.

We compared parameters of the \gls{UNF} inventory
such as activity and decay heat, using the
\gls{PyNE} \cite{scopatz_pyne:_2012}. This tool provides
various functions used
in this work, such as decaying and calculating
the activity and decay heat of the \gls{UNF} inventories.

Ideally, the model should be tested against data
that is not part of the training data. However, given
that the purpose of this model is to allow accessible and
quick depletion calculation for
fuel cycle simulations, the model would suffice if
it predicts the dataset well. In other words, the
goal of the model is to be able to reproduce,
in a continuous manner, the range of fuel depletion
calculations in the database without access to the particular dataset.
This work
is to propose a general concept for implementing
rapid depletion models in fuel cycle simulations.
For creating depletion models for other reactor designs or depletion parameters,
one would simply change the dataset to a set of depletion calculations performed
for a specific reactor design and operational parameters.

We also tested the neural network's performance with the
test set that was set aside prior to the hyperparameter search.



\subsection{Model Implementation in \Cyclus}

The trained model is exported to a file
that can be plugged into external codes. \Cyclus
has a python interface that allows the developer
to design archetypes in python. We developed a reactor
module that behaves similarly to the recipe reactor
but calculates depleted fuel concentration using the
imported model instead of a recipe. The user can vary
individual assembly burnups. The user defines a burnup
and enrichment
matrix for the reactor. The rows are the number
of batches, and the columns are the number of
assemblies in a batch. This reactor module is also
available on Github 
(https://github.com/jbae11/ann\_pwr).

This sort of implementation can be done with
a recipe-based approach for modeling reactor depletion
if the user defines multiple
output recipes. However, the user can only define
the recipe of a batch.
Implementing this trained neural network model will allow the user to vary
burnup and enrichment for individual assemblies, as well
as vary fuel residence time and burnup with reactor
lifetime or simulation time. Such capability will be
useful in simulating \gls{UNF} inventory in the future,
where the burnup of \gls{LWR} fuel will increase
with advanced fuel technology.