\section{Method}

This work follows four steps. First, the
received data with assembly information is
curated for ease of use in training the model.
Second, we train the neural network using Keras and
Scikit-learn, while using an outer loop to
search for the optimized neural network set of hyperparameters.
Third, we use the model to predict the U.S. \gls{UNF}
inventory as specified in the \gls{UDB} and compare
\gls{UNF} inventory metrics such as fissile content
and decay heat.


This work follows four steps:
\begin{enumerate}
\item Data curation
\item Training model (with hyperparameter search)
\item Validation of model
\item Implementing model in \Cyclus.
\end{enumerate}

\subsection{Training and testing dataset}

In order to train a depletion model, a large amount of
depletion data is needed. Given a large database,
we can train the algorithm to predict depleted
nuclear fuel composition.

\subsubsection{Unified Database}
The \gls{UDB} is part of a larger engineering
analysis tool, the \gls{UNF-ST\&DARDS}, developed
by \gls{ORNL} \cite{peterson_used_2013}. The
database ``provides a comprehensive, controlled
source of \gls{SNF} information, including
dry cask attributes, assembly data, economic attributes,
transportation infrastructure attributes, potential future
facility attributes, and federal government radioactive
waste attributes.

We received the database through personal contact
with [] [how we got it and stuff]

list of assumptions from the \gls{UDB-STANDARDS} dataset.
\begin{itemize}
    \item UNF database 
    \item different assembly geometries?
\end{itemize}


Ideally, the data should be generated with exact same reactor
parameters other than burnup and enrichment, such as geometry or
coolant density. However, since
this is preliminary work to prove potential workflow, we
simply used all the \gls{PWR} datasets in the \gls{UDB}.


\subsection{Data Curation}

We curated the raw \gls{UDB} datasets to generate
a cleaner training set. First we only used the 
\gls{PWR} assemblies since \gls{BWR} \gls{UNF} assembly
calculations can vary largely due to coolant density
and reactor design, with regards to two-phase flow. (?!)
We also filtered out the 
`very low' enrichment (< 1.5) and burnup (< 10,000 MWdth/kg)
assemblies to represent more modern \gls{PWR} \gls{UNF}
assembly range.
Also, the SCALE calculation in the \gls{UDB} only tracks 60 isotopes,
and \textasciitilde 3\% of the \gls{UNF} is not accounted for. We
aggregate the isotopes not accounted for as `others'. Lastly,
we curated the database so that the isotopic compositions are 
represented as \% weight from initial uranium mass, to normalize
the dataset.

\begin{equation}
x_i = \frac{m_i}{M_{initU}}
i = \text{isotope}
x_i = \text{\% weight of depleted assembly}
m_i = \text{mass of isotope in depleted assembly in \gls{UDB}}
M_{initU} = \text{Mass of initial uranium in assembly}
\end{equation}


\subsection{Predictive Models}

Isotopic composition of \gls{UNF} is hard to predict due to their
non-linear relationship with reactor parameters. For each
isotope, we plotted isotopic concentration values against
burnup and enrichment to observe the relationship between
burnup, enrichment, and isotope concentration.

[batesman equation analogy]
For example, if the isotope population is mainly determined by
the fission of initial uranium, a linear regression algorithm
can be used to predict the isotope concentration from burnup
(Cs-137 shown in figure \ref{fig:cs_137}).
However, isotopes like plutonium-239 (figure \ref{fig:pu_239}) have multiple, conflicting creation
and destruction terms, making it harder to predict using a
linear regression algorithm. Also, the uranium-235 (figure \ref{fig:u_235}) concentration
depends on both burnup and enrichment, which can make it
hard to predict using a simple linear model.

Due to these complexities, we decided to train an artificial
neural network for our predictive model. We chose
Keras to create and validate the model, as well as scikit-learn
and pandas for data processing and management.

\begin{figure}
    \centering
    \includegraphics[width=\textwidth]{./isos/cs-137_sub.png}
    \caption{Cesium-137 concentration in a \gls{UNF} assembly
             has a linear relationship with assembly burnup.}
    \label{fig:cs_137}
\end{figure}


\begin{figure}
    \centering
    \includegraphics[width=\textwidth]{./isos/pu-239_sub.png}
    \caption{Plutonium-239 composition in a \gls{UNF} assembly
             is not directly proportional to burnup, and is
             not related to initial enrichment.}
    \label{fig:pu_239}
\end{figure}
\begin{figure}
    \centering
    \includegraphics[width=\textwidth]{./isos/u-235_sub.png}
    \caption{Uranium-235 composition in a \gls{UNF} assembly
             somewhat proportional to both enrichment and
             burnup, and is difficult to predict using
             a simple linear regression algorithm.}
    \label{fig:u_235}
\end{figure}


\subsection{Training and Selecting Models}

The input space (feature space) of the model is set to
be burnup (MWdth/kg) and initial enrichment (wt\% U-235).
The output space (target space) of the model is set to
be the individual concentration of the 60 isotopes in the
depleted assembly.




\begin{itemize}
    Training model
    \begin{itemize}
        \item Describe data - feature and target space, number, std deviation etc
        \item Training data curation
    \end{itemize}
    \item normalization of input and output space
    \item saving the model with scalers
    \item hyperparameter search
\end{itemize}






\subsection{Testing Models}
\begin{itemize}
    \item US inventory comparison with UNF STANDARDS
    \item Comparison with SERPENT results
    \begin{itemize}
        \item isotope tracking, assumptions
    \end{itemize}
\end{itemize}
