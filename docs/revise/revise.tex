%        File: revise.tex
%     Created: Wed Oct 27 02:00 PM 2018 P
% Last Change: Wed Oct 27 02:00 PM 2018 P
%

%
% Copyright 2007, 2008, 2009 Elsevier Ltd
%
% This file is part of the 'Elsarticle Bundle'.
% ---------------------------------------------
%
% It may be distributed under the conditions of the LaTeX Project Public
% License, either version 1.2 of this license or (at your option) any
% later version.  The latest version of this license is in
%    http://www.latex-project.org/lppl.txt
% and version 1.2 or later is part of all distributions of LaTeX
% version 1999/12/01 or later.
%
% The list of all files belonging to the 'Elsarticle Bundle' is
% given in the file `manifest.txt'.
%

% Template article for Elsevier's document class `elsarticle'
% with numbered style bibliographic references
% SP 2008/03/01
%
%
%
% $Id: elsarticle-template-num.tex 4 2009-10-24 08:22:58Z rishi $
%
%
%\documentclass[preprint,12pt]{elsarticle}
\documentclass[answers,11pt]{exam}

% \documentclass[preprint,review,12pt]{elsarticle}

% Use the options 1p,twocolumn; 3p; 3p,twocolumn; 5p; or 5p,twocolumn
% for a journal layout:
% \documentclass[final,1p,times]{elsarticle}
% \documentclass[final,1p,times,twocolumn]{elsarticle}
% \documentclass[final,3p,times]{elsarticle}
% \documentclass[final,3p,times,twocolumn]{elsarticle}
% \documentclass[final,5p,times]{elsarticle}
% \documentclass[final,5p,times,twocolumn]{elsarticle}

% if you use PostScript figures in your article
% use the graphics package for simple commands
% \usepackage{graphics}
% or use the graphicx package for more complicated commands
\usepackage{graphicx}
% or use the epsfig package if you prefer to use the old commands
% \usepackage{epsfig}

% The amssymb package provides various useful mathematical symbols
\usepackage{amssymb}
% The amsthm package provides extended theorem environments
% \usepackage{amsthm}
\usepackage{amsmath}

% The lineno packages adds line numbers. Start line numbering with
% \begin{linenumbers}, end it with \end{linenumbers}. Or switch it on
% for the whole article with \linenumbers after \end{frontmatter}.
\usepackage{lineno}

% I like to be in control
\usepackage{placeins}

% natbib.sty is loaded by default. However, natbib options can be
% provided with \biboptions{...} command. Following options are
% valid:

%   round  -  round parentheses are used (default)
%   square -  square brackets are used   [option]
%   curly  -  curly braces are used      {option}
%   angle  -  angle brackets are used    <option>
%   semicolon  -  multiple citations separated by semi-colon
%   colon  - same as semicolon, an earlier confusion
%   comma  -  separated by comma
%   numbers-  selects numerical citations
%   super  -  numerical citations as superscripts
%   sort   -  sorts multiple citations according to order in ref. list
%   sort&compress   -  like sort, but also compresses numerical citations
%   compress - compresses without sorting
%
% \biboptions{comma,round}

% \biboptions{}


% Katy Huff addtions
\usepackage{xspace}
\usepackage{color}

\usepackage{multirow}
\usepackage[hyphens]{url}


\usepackage[acronym,toc]{glossaries}
\include{acros}

\makeglossaries

%\journal{Annals of Nuclear Energy}

\begin{document}

%\begin{frontmatter}

% Title, authors and addresses

% use the tnoteref command within \title for footnotes;
% use the tnotetext command for the associated footnote;
% use the fnref command within \author or \address for footnotes;
% use the fntext command for the associated footnote;
% use the corref command within \author for corresponding author footnotes;
% use the cortext command for the associated footnote;
% use the ead command for the email address,
% and the form \ead[url] for the home page:
%
% \title{Title\tnoteref{label1}}
% \tnotetext[label1]{}
% \author{Name\corref{cor1}\fnref{label2}}
% \ead{email address}
% \ead[url]{home page}
% \fntext[label2]{}
% \cortext[cor1]{}
% \address{Address\fnref{label3}}
% \fntext[label3]{}

\title{Deep Learning Approach to Nuclear Fuel Transmutation in a Fuel
Cycle Simulator\\
        \large Response to Review Comments}
\author{Jin Whan Bae, Andrei Rykhlevskii, Gwendolyn Chee, Kathryn D. Huff}

% use optional labels to link authors explicitly to addresses:
% \author[label1,label2]{<author name>}
% \address[label1]{<address>}
% \address[label2]{<address>}


%\author[uiuc]{Kathryn Huff}
%        \ead{kdhuff@illinois.edu}
%  \address[uiuc]{Department of Nuclear, Plasma, and Radiological Engineering,
%        118 Talbot Laboratory, MC 234, Universicy of Illinois at
%        Urbana-Champaign, Urbana, IL 61801}
%
% \end{frontmatter}
\maketitle
\section*{Review General Response}
We would like to thank the reviewers for their assessment of
this paper. Your comments have resulted in changes which certainly improved the 
paper. 


\section*{Reviewer 1}
\begin{questions}

        %---------------------------------------------------------------------
        \question This is a really neat paper. The value of this as an open source tool to do transmutation analyses is profound. I am a bit concerned about people applying such a tool outside of its range of applicability... and what would happen in that case. Perhaps the authors could comment.

        \begin{solution}
        	Thank you for your valuable input. We do acknowledge that it is a tricky
            situation to have a neural network ``mimic'' export control software, and 
            there has been a lot of discussion on whether we should control such
            tools. We believe that the designed neural network for this work does not have
            any malicious applicability, since it is limited to \gls{PWR} depletion
            using burnup and enrichment. However, given the security aspect of nuclear
            engineering research, a conversation about how certain algorithms should be
            treated would be an interesting (and necessary) discussion in the community.

            We added the following paragraph at the end of the discussion section:
                        
            Another interesting topic of discussion is the possible
            issues that arise from openly distributing a tool that `mimics'
            an export control software, in this case, SCALE/ORIGEN. We believe
            that the designed neural network for this work does not have any
            potential to be used other than \gls{PWR} depletion. However, given
            the recent increase in the application of machine learning
            for nuclear engineering, a conversation about how certain models should
            be controlled would be an interesting (and necessary) discussion in the community.

        \end{solution}


        \section*{Reviewer 2}


        %---------------------------------------------------------------------
        \question O1. page 4, line 49 : delete the 'Fuel' due to duplication
                  2. page 34, line 47 and 56; add the publication years in reference 9 and 10


        \begin{solution}
		Thank you for catching these. We made the appropriate changes.

        \end{solution}




        %---------------------------------------------------------------------
\end{questions}
\bibliographystyle{unsrt}
\bibliography{revise}
\end{document}

  %
  % End of file `elsarticle-template-num.tex'.
